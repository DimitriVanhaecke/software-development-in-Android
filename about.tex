\chapter{About this course}

\section{Introduction}
This course is an introduction into the wonderful world of Android.
The aim of this course is not to turn you into a specialized Android developer, but to give an introduction to the general concepts of the Android framework and provide the tools you need to find your way in this fragmented world. 

Starting from first semester of 2018 we will adopt Kotlin as the default language for developing Android applications. 

\section{ECTS}
ECTS is an acronym for European Credit Transfer and Accumulation System.
This stipulates every final attainment level you should have when finishing this course.
These can be found \href{https://bamaflexweb.hogent.be/BMFUIDetailxOLOD.aspx?a=98522\&b=5\&c=1}{here}. 

\section{Course Material}
The course is taught using the following materials:

\begin{itemize}
	\item These course notes
	\item Book \textit{Kotlin for Android Developers} \cite{Leiva2018}
	\item Book  \textit{The Busy Coder's Guide to Android Development} \cite{murphymarkl.2017}
	\item External resources and references (these course notes, Chamilo links \dots)
	\item TODO: boeken toevoegen die we gebruikt hebben
\end{itemize}

\section{Course organisation}
This course consists of 12 lessons.
The first part of each lesson will cover the theoretical aspects and the second part will put theory into practice.
Normally you will not be able to complete the exercises in one hour so it is up to you to complete these exercises at home.
If you encounter any difficulties  you are welcome to ask questions in the next lesson or via the forum on Chamilo.
We don't encourage sending personal e-mail: you are not alone in having difficulties with certain topics.
E-mail is not the best way for us to help you.

Think carefully when you write down your question: make it reproducible. Follow these guidelines: \href{https://stackoverflow.com/help/how-to-ask}{https://stackoverflow.com/help/how-to-ask}

\subsection{Covered Topics}
This course will roughly cover the following topics:

\begin{enumerate}
	\item History of  Android and Intro Kotlin
	\item Activities + layout files 
	\item Structure of an  app, Activity lifecycle, Resources, buildproces
	\item Fragments + UI components and layouts 
	\item Intents, BroadcastReceivers (EventBus) and Parcable 
	\item recyclerview + cardview 
	\item Android coding practices  + common Android software architectures  
	\item Persistentence in Android
	\item Networking
	\item Testing in Android
	\item Composer for android testing
	\item Memory leaks 
	\item Firebase 
\end{enumerate}

\section{Prior knowledge}
This course assumes knowledge of the Java programming language.
You do not need to know everything, but you need to understand the following concepts:

\begin{itemize}
	\item Language fundamentals (flow control, etc.)
	\item Classes and objects
	\item Methods and data members
	\item Public, private, and protected
	\item Static and instance scope
	\item Exceptions
	\item Threads
	\item Collections
	\item Generics
	\item File I/O
	\item Reflection
	\item Interfaces
\end{itemize}

\section{Examination}
Both for the first as for the second exam period, you will be examined via an oral exam.
You will be handed a set of questions and  you will have time to prepare them.
Some of the questions will be regarding the exercises you have made during the year, so you are obliged to have made them all.
\textbf{If your question is one regarding an exercise you have not made, you will receive a 0 on that question.}

Working together on the exercises is encouraged, but make sure that you have written all the code yourself.
You will get some hard questions and it is very difficult to explain code that isn't yours.
Document your code well and make sure you remember why you have written it the way you have.

