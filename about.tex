\chapter{About this course}

\section{Introduction}
This course is an introduction into the wonderful world of Android. The aim of this course is not to generate specialized Android developers, but to introduce you to the general concepts of the Android framework and provide you to tools to find your way in this fragmented world. 

Starting from first semester of 2018 we will adopt Kotlin as the default language for developing Android applications. 

\section{ECTS}
ECTS is an acronym for European Credit Transfer and Accumulation System. This stipulates every final attainment level you should have when finishing this course. These can be found \href{here}{https://bamaflexweb.hogent.be/BMFUIDetailxOLOD.aspx?a=98522\&b=5\&c=1}. 

\section{Course Material}
The course is taught using the following materials:

\begin{itemize}
	\item These course notes
	\item Book \textit{Kotlin for Android Developers} \cite{Leiva2018}
	\item Book  \textit{The Busy Coder's Guide to Android Development} \cite{murphymarkl.2017}
	\item External resources and references (these course notes, Chamilo links \dots)
	\item The usefull post from ProdAndroidDev \url{https://proandroiddev.com/}
	\item TODO: boeken toevoegen die we gebruikt hebben
\end{itemize}

\section{Course organisation}
The lessons are organized in 12 lessons, where we will go over the theoretical aspects in the first part of the lesson, where the second part will be used to put things learned into practice. Normally you will not be able to complete the exercises in one hour so it is up to you to complete the exercises at home. If you encounter any difficulties  you are welcome to ask questions in the next lesson or via the forum on Chamilo. We don't encourage sending personal email: you are
not alone in having difficulties with certain topics. Email is not the best way for us to help you.

When asking questions, take care to format your questions: make it reproducible. Follow these guidelines: \href{https://stackoverflow.com/help/how-to-ask}{https://stackoverflow.com/help/how-to-ask}

\subsection{Covered Topics}
This course will roughly cover the following topics:

\begin{enumerate}
	\item History of  Android and Intro Kotlin
	\item Activities + layout files 
	\item Structure of an  app, Activity lifecycle, Resources, buildproces
	\item Fragments + UI components and layouts 
	\item Intents, BroadcastReceivers (EventBus) and Parcable 
	\item recyclerview + cardview 
	\item Android coding practices  + common Android software architectures  
	\item Persistentence in Android
	\item Networking
	\item Testing in Android
	\item Composer for android testing
	\item Memory leaks 
	\item Firebase 
\end{enumerate}

\section{Prior knowledge}
This course assumes that you know Java at this point. If you do not, you will need to learn Java before you go much further. You do not need to know everything about Java, as Java is vast. Rather, focus on:

\begin{itemize}
	\item Language fundamentals (flow control, etc.)
	\item Classes and objects
	\item Methods and data members
	\item Public, private, and protected
	\item Static and instance scope
	\item Exceptions
	\item Threads
	\item Collections
	\item Generics
	\item File I/O
	\item Reflection
	\item Interfaces
\end{itemize}

\section{Examination}
Both for the first as for the second exam period, you will be examined via an oral exam. You will be handed a set of questions and  you will have some time to prepare them. Some of the questions will be regarding the exercises you have made during the year, so you are obliged to have made all the exercises. \textbf{If your question is one regarding an exercise you have not made, you will receive a 0 on that question.}

Working together on the exercises is encouraged, but make sure that you have programmed and written the code yourself. You will get some hard questions and it is very difficult to answer when you have not written the code yourself.
Document your code well! Make sure you remember why you have written something the way you have.

